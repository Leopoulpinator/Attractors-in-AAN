\documentclass[12pt,a4paper]{article}
\usepackage{hyperref}
\usepackage{fancyhdr}
\usepackage[a4paper,hmargin=19mm,vmargin=25mm]{geometry}
\usepackage{nameref}
\usepackage[symbol]{footmisc}
\usepackage{minitoc}
\usepackage{color}
\usepackage{amsmath}
\setcounter{tocdepth}{3}

\newcommand{\HRule}{\rule{\linewidth}{0.5mm}}
\renewcommand{\thefootnote}{\fnsymbol{footnote}}
\renewcommand{\contentsname}{Table des matières}

% Macros relatives à la traduction de PH avec arcs neutralisants vers PH à k-priorités fixes

% Macros générales
%\newcommand{\ie}{\textit{i.e.} }
\newcommand{\segm}[2]{\llbracket #1; #2 \rrbracket}
%\newcommand{\f}[1]{\mathsf{#1}}

% Notations générales pour PH
\newcommand{\PH}{\mathcal{PH}}
%\newcommand{\PHs}{\mathcal{S}}
\newcommand{\PHs}{\Sigma}
%\newcommand{\PHp}{\mathcal{P}}
\newcommand{\PHp}{\textcolor{red}{\mathcal{P}}}
%\newcommand{\PHproc}{\mathcal{P}}
\newcommand{\PHproc}{\mathbf{Proc}}
\newcommand{\Proc}{\PHproc}
\newcommand{\PHh}{\mathcal{H}}
\newcommand{\PHa}{\PHh}
%\newcommand{\PHa}{\mathcal{A}}
\newcommand{\PHl}{\mathcal{L}}
\newcommand{\PHn}{\mathcal{N}}
\newcommand{\PHt}{\mathcal{H}_{t}} % Ensemble des action du T-PH avec des actions temporisées

\newcommand{\PHhitter}{\mathsf{hitter}}
\newcommand{\PHtarget}{\mathsf{target}}
\newcommand{\PHbounce}{\mathsf{bounce}}
%\newcommand{\PHsort}{\Sigma}
\newcommand{\PHsort}{\PHs}
%State of PH
\newcommand{\PHst}{\zeta}

%Automata Network
\newcommand{\AN}{\mathcal{AAN}}
\newcommand{\TAN}{T-\mathcal{AN}}
\newcommand{\ANsort}{\Sigma}
\newcommand{\ANbound}[1]{\operatorname{b}(#1)}
\newcommand{\ANstate}{\mathcal{S}}
\newcommand{\ANtrans}{\mathcal{T}}
\newcommand{\ANst}{\zeta}
%\newcommand{\LS}{\textbf{LS}}
\newcommand{\LS}{\mathcal{LS}}
\newcommand{\ANgt}{\rightarrow}
\newcommand{\ANgtu}[1]{\ANgt_{#1}}
\newcommand{\origin}[1]{\mathsf{ori}(#1)}
\newcommand{\condition}[1]{\mathsf{cond}(#1)}
\newcommand{\destination}[1]{\mathsf{dest}(#1)}
%\newcommand{\ANuexp}[1]{U^{#1}}
\newcommand{\ANuexpsymbol}{\,\mathcal{U}}
\newcommand{\ANuexp}[1]{\ANuexpsymbol^{#1}}
\newcommand{\U}{\ANuexpsymbol}
\newcommand{\Ua}{\ANuexp{\mathsf{asyn}}}
\newcommand{\Us}{\ANuexp{\mathsf{syn}}}
\newcommand{\ANglobalstate}[1]{\langle #1 \rangle}
\newcommand{\ANltsymb}[4]{#1 \overset{#2}{#4} #3}
\newcommand{\ANlt}[3]{\ANltsymb{#1}{#2}{#3}{\rightarrow}}
\newcommand{\ANlts}[3]{\ANltsymb{#1}{\{#2\}}{#3}{\longrightarrow}}
\newcommand{\ANltsc}[3]{\textcolor{commentgray}{\ANlts{#1}{#2}{#3}}}
\newcommand{\aiellaj}{\ANlt{a_i}{\ell}{a_j}}
\newcommand{\trace}{\mathsf{trace}}
\newcommand{\successiverep}{\mathsf{sr}}

%\newcommand{\PHfrappeur}{\mathsf{frappeur}}
%\newcommand{\PHcible}{\mathsf{cible}}
%\newcommand{\PHbond}{\mathsf{bond}}
%\newcommand{\PHsorte}{\mathsf{sorte}}
%\newcommand{\PHbloquant}{\mathsf{bloquante}}
%\newcommand{\PHbloque}{\mathsf{bloquee}}

%\newcommand{\PHfrappeR}{\textcolor{red}{\rightarrow}}
%\newcommand{\PHmonte}{\textcolor{red}{\Rsh}}

\newcommand{\PHhitA}{\rightarrow}
\newcommand{\PHhitB}{\Rsh}
%\newcommand{\PHfrappe}[3]{\mbox{$#1\PHhitA#2\PHhitB#3$}}
%\newcommand{\PHfrappebond}[2]{\mbox{$#1\PHhitB#2$}}
\newcommand{\PHhit}[3]{#1\PHhitA#2\PHhitB#3}
\newcommand{\PHfrappe}{\PHhit}
\newcommand{\PHhbounce}[2]{#1\PHhitB#2}
\newcommand{\PHobj}[2]{\mbox{$#1\PHhitB^*\!#2$}}
\newcommand{\PHobjectif}{\PHobj}
\newcommand{\PHconcat}{::}
%\newcommand{\PHneutralise}{\rtimes}
\def\Sce{\mathbf{Sce}}

% Actions plurielles
\newcommand{\PHhitmultsymbol}{\rightarrowtail}
\newcommand{\PHhitmult}[2]{\mbox{$#1 \PHhitmultsymbol #2$}}
\newcommand{\PHfrappemult}{\PHhitmult}
\newcommand{\PHfrappemults}[2]{\PHhitmult{\{#1\}}{\{#2\}}}

%Action plurielle avec délai
\newcommand{\PHhitdelayB}{\Rsh}
\newcommand{\PHhitdelayA}[1]{\xrightarrow{#1} }
\newcommand{\PHhitdelay}[4]{#1\PHhitdelayA{#2} #3 \PHhitdelayB #4}
\newcommand{\PHfrappedelay}{\PHhitdelay}

\def\PHget#1#2{{#1[#2]}}
%\newcommand{\PHchange}[2]{#1\langle #2 \rangle}
%\newcommand{\PHchange}[2]{(#1 \Lleftarrow #2)}
%\newcommand{\PHarcn}[2]{\mbox{$#1\PHneutralise#2$}}
\newcommand{\PHplay}{\cdot}

\newcommand{\PHstate}[1]{\mbox{$\langle #1 \rangle$}}
\newcommand{\PHetat}{\PHstate}

\def\supp{\mathsf{support}}
\def\first{\mathsf{first}}
\def\last{\mathsf{last}}

\def\DNtrans{\rightarrow_{ADN}}
\def\DNdef{(\mathbb F, \langle f^1, \dots, f^n\rangle)}
\def\DNdep{\mathsf{dep}}
\def\PHPtrans{\rightarrow_{PH}}
\def\get#1#2{#1[{#2}]}
\def\encodeF#1{\mathbf{#1}}
\def\toPH{\encodeF{PH}}
\def\card#1{|#1|}
\def\decode#1{\llbracket#1\rrbracket}
\def\encode#1{\llparenthesis#1\rrparenthesis}
\def\Hits{\PHa}
\def\hit{\PHhit}
\def\play{\cdot}

\def\Pint{\textsc{PINT}}

\newcommand{\sN}{\mathbb{N}}

\usepackage{ifthen}

\newcommand{\currentScope}{}
\newcommand{\currentSort}{}
\newcommand{\currentSortLabel}{}
\newcommand{\currentAlign}{}
\newcommand{\currentSize}{}

\newcounter{la}
\newcommand{\TSetSortLabel}[2]{
  \expandafter\repcommand\expandafter{\csname TUserSort@#1\endcsname}{#2}
}
\newcommand{\TSort}[4]{
  \renewcommand{\currentScope}{#1}
  \renewcommand{\currentSort}{#2}
  \renewcommand{\currentSize}{#3}
  \renewcommand{\currentAlign}{#4}
  \ifcsname TUserSort@\currentSort\endcsname
    \renewcommand{\currentSortLabel}{\csname TUserSort@\currentSort\endcsname}
  \else
    \renewcommand{\currentSortLabel}{\currentSort}
  \fi
  \begin{scope}[shift={\currentScope}]
  \ifthenelse{\equal{\currentAlign}{l}}{
    \filldraw[process box] (-0.5,-0.5) rectangle (0.5,\currentSize-0.5);
    \node[sort] at (-0.2,\currentSize-0.4) {\currentSortLabel};
   }{\ifthenelse{\equal{\currentAlign}{r}}{
     \filldraw[process box] (-0.5,-0.5) rectangle (0.5,\currentSize-0.5);
     \node[sort] at (0.2,\currentSize-0.4) {\currentSortLabel};
   }{
    \filldraw[process box] (-0.5,-0.5) rectangle (\currentSize-0.5,0.5);
    \ifthenelse{\equal{\currentAlign}{t}}{
      \node[sort,anchor=east] at (-0.3,0.2) {\currentSortLabel};
    }{
      \node[sort] at (-0.6,-0.2) {\currentSortLabel};
    }
   }}
  \setcounter{la}{\currentSize}
  \addtocounter{la}{-1}
  \foreach \i in {0,...,\value{la}} {
    \TProc{\i}
  }
  \end{scope}
}

\newcommand{\TTickProc}[2]{ % pos, label
  \ifthenelse{\equal{\currentAlign}{l}}{
    \draw[tick] (-0.6,#1) -- (-0.4,#1);
    \node[tick label, anchor=east] at (-0.55,#1) {#2};
   }{\ifthenelse{\equal{\currentAlign}{r}}{
    \draw[tick] (0.6,#1) -- (0.4,#1);
    \node[tick label, anchor=west] at (0.55,#1) {#2};
   }{
    \ifthenelse{\equal{\currentAlign}{t}}{
      \draw[tick] (#1,0.6) -- (#1,0.4);
      \node[tick label, anchor=south] at (#1,0.55) {#2};
    }{
      \draw[tick] (#1,-0.6) -- (#1,-0.4);
      \node[tick label, anchor=north] at (#1,-0.55) {#2};
    }
   }}
}
\newcommand{\TSetTick}[3]{
  \expandafter\repcommand\expandafter{\csname TUserTick@#1_#2\endcsname}{#3}
}

\newcommand{\myProc}[3]{
  \ifcsname TUserTick@\currentSort_#1\endcsname
    \TTickProc{#1}{\csname TUserTick@\currentSort_#1\endcsname}
  \else
    \TTickProc{#1}{#1}
  \fi
  \ifthenelse{\equal{\currentAlign}{l}\or\equal{\currentAlign}{r}}{
    \node[#2] (\currentSort_#1) at (0,#1) {#3};
  }{
    \node[#2] (\currentSort_#1) at (#1,0) {#3};
  }
}
\newcommand{\TSetProcStyle}[2]{
  \expandafter\repcommand\expandafter{\csname TUserProcStyle@#1\endcsname}{#2}
}
\newcommand{\TProc}[1]{
  \ifcsname TUserProcStyle@\currentSort_#1\endcsname
    \myProc{#1}{\csname TUserProcStyle@\currentSort_#1\endcsname}{}
  \else
    \myProc{#1}{process}{}
  \fi
}

\newcommand{\repcommand}[2]{
  \providecommand{#1}{#2}
  \renewcommand{#1}{#2}
}
\newcommand{\THit}[5]{
  \path[hit] (#1) edge[#2] (#3#4);
  \expandafter\repcommand\expandafter{\csname TBounce@#3@#5\endcsname}{#4}
}
\newcommand{\TBounce}[4]{
  (#1\csname TBounce@#1@#3\endcsname) edge[#2] (#3#4)
}

%\newcommand{\TState}[1]{
%  \foreach \proc in {#1} {
%    \node[current process] (\proc) at (\proc.center) {};
%  }
%}

\newcommand{\TState}[1]{
  \foreach \proc in {#1} {
        \node[current process] (\proc) at (\proc.center) {};
  };
}
\newcommand{\TCoopHit}[6]{
  \node[#2, apdot] at (#3) {};
  \foreach \proc in {#1} {
    \draw[#2,-] (#3) edge (\proc);
  }
  \path[hit] (#3) edge[#2] (#4#5);
  \expandafter\repcommand\expandafter{\csname TBounce@#4@#6\endcsname}{#5}
}

% ex : \TAction{c_1}{a_1.west}{a_0.north west}{}{right}
% #1 = frappeur
% #2 = cible
% #3 = bond
% #4 = style frappe
% #5 = style bond
\newcommand{\TAction}[5]{
  \THit{#1}{#4}{#2}{}{#3}
  \path[bounce, bend #5=50] \TBounce{#2}{}{#3}{};
}

% ex : \TActionPlur{f_1, c_0}{a_0.west}{a_1.south west}{}{3.5,2.5}{left}
% #1 = frappeur
% #2 = cible
% #3 = bond
% #4 = style frappe
% #5 = coordonnées point central
% #6 = direction bond
\newcommand{\TActionPlur}[6]{
  \TCoopHit{#1}{#4}{#5}{#2}{}{#3}
  \path[bounce, bend #6=50] \TBounce{#2}{}{#3}{};
}

% Styles TikZ et couleurs personnalisées

\usepackage{tikz}

\newdimen\pgfex
\newdimen\pgfem
\usetikzlibrary{arrows,shapes,shadows,scopes}
\usetikzlibrary{positioning}
\usetikzlibrary{matrix}
\usetikzlibrary{decorations.text}
\usetikzlibrary{decorations.pathmorphing}
\usetikzlibrary{arrows,shapes}

\definecolor{lightgray}{rgb}{0.8,0.8,0.8}
\definecolor{lightgrey}{rgb}{0.8,0.8,0.8}

\definecolor{lightred}{rgb}{1,0.8,0.8}
\definecolor{lightgreen}{rgb}{0.7,1,0.7}
\definecolor{darkgreen}{rgb}{0,0.5,0}
\definecolor{darkblue}{rgb}{0,0,0.5}
\definecolor{darkyellow}{rgb}{0.5,0.5,0}
\definecolor{lightyellow}{rgb}{1,1,0.6}
\definecolor{darkcyan}{rgb}{0,0.6,0.6}
\definecolor{lightcyan}{rgb}{0.6,1,1}
\definecolor{darkorange}{rgb}{0.8,0.2,0}
\definecolor{notsodarkred}{rgb}{0.8,0,0}

\definecolor{notsodarkgreen}{rgb}{0,0.7,0}

%\definecolor{coloract}{rgb}{0,1,0}
%\definecolor{colorinh}{rgb}{1,0,0}
\colorlet{coloract}{darkgreen}
\colorlet{colorinh}{red}
\colorlet{coloractgray}{lightgreen}
\colorlet{colorinhgray}{lightred}
\colorlet{colorinf}{darkgray}
\colorlet{coloractgray}{lightgreen}
\colorlet{colorinhgray}{lightred}

\colorlet{colorgray}{lightgray}
\colorlet{colorhl}{blue}


\tikzstyle{boxed ph}=[]
\tikzstyle{sort}=[fill=lightgray, rounded corners, draw=black]
\tikzstyle{process}=[circle,draw,minimum size=15pt,fill=white,font=\footnotesize,inner sep=1pt]
%\tikzstyle{black process}=[process, draw=blue, fill=red,text=black,font=\bfseries]
\tikzstyle{gray process}=[process, draw=black, fill=lightgray]
\tikzstyle{highlighted process}=[current process, fill=gray]
\tikzstyle{process box}=[fill=none,draw=black,rounded corners]
%\tikzstyle{current process}=[process, draw=black, fill=lightgray]
\tikzstyle{current process}=[process,fill=lightgray]
\tikzstyle{hl process}=[process,fill=blue!30]
\tikzstyle{tick label}=[font=\footnotesize]
\tikzstyle{tick}=[densely dotted] %-
\tikzstyle{hit}=[->,>=angle 45]
\tikzstyle{selfhit}=[min distance=50pt,curve to]
\tikzstyle{bounce}=[densely dotted,>=stealth',->]
\tikzstyle{ulhit}=[draw=lightgray,fill=lightgray]
\tikzstyle{pulhit}=[fill=lightgray]
\tikzstyle{bulhit}=[draw=lightgray]
\tikzstyle{hl}=[very thick,colorhl]
\tikzstyle{hlb}=[very thick]
\tikzstyle{hlhit}=[hl]
%\tikzstyle{hl2}=[hl]
%\tikzstyle{nohl}=[font=\normalfont,thin]

\tikzstyle{update}=[draw,->,dashed,shorten >=.7cm,shorten <=.7cm]

\tikzstyle{unprio}=[draw,thin]%[double]
%\tikzstyle{prio}=[draw,thick,-stealth]%[double]
\tikzstyle{prio}=[draw,-stealth,double]

\tikzstyle{hitless graph}=[every edge/.style={draw=red,-}]

\tikzstyle{aS}=[every edge/.style={draw,->,>=stealth}]
\tikzstyle{Asol}=[draw,circle,minimum size=5pt,inner sep=0,node distance=1cm]
\tikzstyle{Aproc}=[draw,node distance=1.2cm]
\tikzstyle{Aobj}=[node distance=1.5cm]
\tikzstyle{Anos}=[font=\Large]

\tikzstyle{AsolPrio}=[Asol,double]
\tikzstyle{AprocPrio}=[Aproc,double]
\tikzstyle{aSPrio}=[aS,double]

\colorlet{colorhlwarn}{notsodarkred}
\colorlet{colorhlwarnbg}{lightred}
\tikzstyle{Ahl}=[very thick,fill=colorhlwarnbg,draw=colorhlwarn,text=colorhlwarn]
\tikzstyle{Ahledge}=[very thick,double=colorhlwarnbg,draw=colorhlwarn,color=colorhlwarn]





%\definecolor{darkred}{rgb}{0.5,0,0}



\tikzstyle{grn}=[every node/.style={circle,draw=black,outer sep=2pt,minimum
                size=15pt,text=black}, node distance=1.5cm, ->]
\tikzstyle{inh}=[>=|,-|,draw=colorinh,thick, text=black,label]
\tikzstyle{act}=[->,>=triangle 60,draw=coloract,thick,color=coloract]
\tikzstyle{inhgray}=[>=|,-|,draw=colorinhgray,thick, text=black,label]
\tikzstyle{actgray}=[->,>=triangle 60,draw=coloractgray,thick,color=coloractgray]
\tikzstyle{inf}=[->,draw=colorinf,thick,color=colorinf]
%\tikzstyle{elabel}=[fill=none, above=-1pt, sloped,text=black, minimum size=10pt, outer sep=0, font=\scriptsize,draw=none]
\tikzstyle{elabel}=[fill=none,text=black, above=-2pt,%sloped,
minimum size=10pt, outer sep=0, font=\scriptsize, draw=none]
%\tikzstyle{elabel}=[]


\tikzstyle{plot}=[every path/.style={-}]
\tikzstyle{axe}=[black,->,>=stealth']
\tikzstyle{ticks}=[font=\scriptsize,every node/.style={black}]
\tikzstyle{mean}=[thick]
\tikzstyle{interval}=[line width=5pt,red,draw opacity=0.7]
%\definecolor{lightred}{rgb}{1,0.3,0.3}

%\tikzstyle{hl}=[yellow]
%\tikzstyle{hl2}=[orange]

%\tikzstyle{every matrix}=[ampersand replacement=\&]
%\tikzstyle{shorthandoff}=[]
%\tikzstyle{shorthandon}=[]
\tikzstyle{objective}=[process,very thick,fill=yellow!50]

\tikzstyle{coopupdate}=[-stealth,decorate,decoration={zigzag,amplitude=1.5pt,post=lineto,post length=.3cm,pre=lineto,pre length=.3cm}]

\tikzstyle{labelprio}=[circle, fill=blue!30, inner sep=0pt, minimum size=13pt]
\tikzstyle{labelprio1}=[labelprio]
\tikzstyle{labelprio2}=[labelprio, fill=red!60]
\tikzstyle{labelprio3}=[labelprio, fill=orange!50]
\tikzstyle{labelprio4}=[labelprio, fill=brown!50]

\tikzstyle{labelstocha}=[rectangle, rounded corners=4pt]

\tikzstyle{andot}=[circle, fill=black, inner sep=1.2pt, draw=transparent]
\tikzstyle{anligne}=[thick]

\tikzstyle{apdot}=[andot] %[circle, fill=black, draw=black, inner sep=1]
\tikzstyle{apdotsimple}=[] %[circle, fill=black, draw=black, inner sep=1]

% Figure de résumé des liens entre les formalismes
\tikzstyle{equiv-externe}=[thick, rounded corners, draw=gray, fill=gray!10, align=center,
  inner sep=8]

% label pour les délais des actions 
 \tikzstyle{labeldelai1}=[circle, fill=red!60, inner sep=0pt, minimum size=8pt]
  \tikzstyle{labeldelai2}=[circle, fill=blue!30, inner sep=0pt, minimum size=8pt]
  \tikzstyle{labeldelai3}=[circle, fill=brown!50, inner sep=0pt, minimum size=8pt]
  \tikzstyle{labeldelai4}=[circle, fill=green!50, inner sep=0pt, minimum size=8pt]
  
  % Automata Networks:
\tikzstyle{local transitions}=[->,>=latex',thick,bend left=30,
               every node/.style={fill=white,inner sep=1pt,outer sep=1pt}]
\tikzstyle{reach}=[fill=lightgray,ellipse]

\begin{document}

\begin{titlepage}
    \begin{center}
        \textsc{\Large Rapport de stage de L3}\\

    \vspace{2.5cm}

        \HRule\\
        {\huge\scshape Analyse de la dynamique des modèles biologiques par programmation logique\par}
        \HRule\\

    \vspace{3cm}

        \large
        Léo-Paul \textsc{Delsaux}\footnote[2]{ENS de Lyon}\\

    \vspace{1cm}

        \large
        \emph{Encadrant du stage} \\
        Maxime \textsc{Folschette}\footnote[3]{Équipe Bio-Computing, labortaoire CRIStAL, CNRS de Lille}\\

    \vfill

        {\large Juin-Août 2022}
    \end{center}
\end{titlepage}

\addtocontents{toc}{\protect\hypertarget{toc}{}}

\fancypagestyle{plain}{%
\fancyhf{}
\cfoot{\hyperlink{toc}{\thepage}}%
\renewcommand{\headrulewidth}{0pt}
\renewcommand{\footrulewidth}{0pt}}
\pagestyle{plain}%

\newpage
\tableofcontents

\newpage
\section{Introduction}
La biologie s'intéresse à l'étude des systèmes vivants ou composant le vivant (gènes, cellules, écosystèmes...). La bioinformatique est l'utilisation de méthodes informatiques pour aider la biologie. Un des domaines de la 
bioinformatique s'apelle la \emph{biologie des systèmes} et consiste à proposer des modèles et des méthodes informatiques et mathématiques pour représenter et étudier de tels systèmes.\\ \\
Ainsi, des études biologiques permettent de déterminer comment certains gènes interagissent entre eux, ce qui peut être représenté mathématiquement sous la forme d'un graphe. De plus, ces interactions peuvent être vues comme des 
transitions entre les sommets de notre graphe, ce qui se rapproche alors plus d'un automate. La représentation du modèle étudié lors de ce stage est explicité dans le chapitre Modèle.\\ \\
L'étude de la dynamique de systèmes biologiques lève plusieurs problèmes tels que l'identification d'attracteurs, les bifurcations ou encore la connexité entre deux états globaux. Dans ce rapport je vais discuter de la recherche 
d'attracteurs dans des réseaux d'automates asynchrones : il s'agit d'un ensemble d'états duquel on ne peut pas s'échapper et minimal au sens de l'inclusion.\\ \\
Answer Set Programming (= ASP) est un paradigme de programmation logique particulièrement efficace pour résoudre des problèmes combinatoires. Je n'ai travaillé qu'avec ce langage (et un peu de scripting python) 
durant mon stage afin de chercher les attracteurs parmi le graphe des états globaux de notre réseau.\\ \\
Ce qui sera étudié dans ce rapport ne le sera que partiellement : on n'entrera pas en détail dans le code présenté, mais on se penchera simplement sur les grandes lignes et sur les points techniques qu'il arbore.

\section{Answer Set Programming}
L'Answer Set Programming (ASP) est un paradigme de programmation logique comparable à Prolog. Ces dernières décennies, ASP s'est trouvé être puissant pour traiter des modèles biologiques, permettant de parcourir un 
grand nombre de configurations rapidement. ASP peut énumérer facilement les ensembles solutions à un problème qu'on lui encode : c'est un paradigme très efficace pour la combinatoire. Nous allons ici présenter brièvement 
son fonctionnement, et plus précisément, les outils qui ont été utiles pour mon stage.
\subsection{Termes}
ASP fonctionne à l'aide de déclarations de faits, de règles, de prédicats contenant un ou plusieurs arguments tels que : \emph{\textcolor{blue}{parentOf(jenny, charles)}}, ce qui nécessite tout d'abord que j'introduise la notion 
de \textbf{terme} (ou \textbf{atome}). On définit un terme de la manière suivante : 
\begin{enumerate}
	\item terme simple :
		\begin{itemize}
		\item un entier relatif (0, -35, 42)
		\item une constante qui démarre par une lettre minuscule(v0us, lis35, quatre, ex3mpl35)
		\item une chaîne de caractères encadrée de guillemets ("ma\_chaine\_de\_caracteres", "42")
		\item une variable identique grammaticalement aux constantes, mais dont la première lettre est majuscule (N0u5, S0mm35, D35, V4r14bl35)
		\item un (ou plusieurs) underscore symbolisant une variable sans nom (\_, \_\_)
		\end{itemize}
	\item une fonction (qu'on appellera également prédicat) est de la forme : constante($t_1$, $t_2$, ..., $t_k$) avec un nombre de termes (qui sont les arguments) fini qui peut être nul (f(23), je\_suis\_une\_fonction(42, 
	true), et\_moi\_aussi(oui)).\\
	Ces fonctions n'ont pas vocation à calculer, mais uniquement à déclarer des objets avec les arguments souhaités comme étant vrais.
	\item un tuple de la forme : ($t_1$, $t_2$, ..., $t_k$) avec un nombre de termes fini qui ne peut être nul ((37, oui), (), (1, h(-4), 0, 1)).
\end{enumerate}
\subsection{Règles et modèles d'un programme}
Un \emph{programme d'ensemble solutions ( = answer set program)} est un nombre fini de règles de la forme :

\begin{center}
$a_0$ :- $a_1,...,a_m,not \: a_{m+1},...,not \: a_n.$
\end{center}
avec $n\geq m\geq m\geq 0$. $a_0$ est un atome ou $\bot$ (Bottom/le Faux) et représente ici la \textbf{tête} de la règle. Les $a_1,...,a_n$ sont des atomes et représente quant à eux le \textbf{corps} de la règle, et le symbole $"not"$ 
représente la négation par l'échec. Cette règle se lit intuitivement : si les atomes $a_1,..,a_m$ sont tous vrais et qu'aucun des atomes $a_{m+1},...,a_n$ n'est vrai, alors $a_0$ est vrai.\\ \\
Si $n=m=0$, cela signifie que $a_0$ est vrai. Dans ce cas, on parle d'un fait, et on ne doit pas renseigner ":-". D'une autre part, si $a_0=\bot$, on parle de contrainte : comme $\bot$ ne peut jamais être vrai, si le corps de la 
règle est vrai, cela 
invalide la solution actuelle. On ne renseigne pas $\bot$ pour la tête, on laisse une tête vide au niveau du code.\\ \\
Cette notion de "solution actuelle" se formalise de la façon suivante : on parle d'une \textbf{interprétation} $I$ en tant qu'ensemble fini d'atomes propositionnels. Une règle $r$ définie comme ci-dessus est \emph{vraie dans I} si 
et seulement si :

\begin{center}
$\left\{a_1,...,a_m\right\}\subseteq I \wedge \left\{a_{m+1},...,a_n\right\} \cap I = \emptyset \Rightarrow a_0 \in I$
\end{center}

Si toutes les règles d'un programme $P$ sont vraies dans une même interprétation $I$ et que $I$ est maximale, alors on dit que $I$ est un \textbf{modèle} de $P$. Lorsque l'on déclare un problème $P$ en ASP, le solveur nous renverra en sortie tout 
les modèles possibles pour $P$.
\subsection{Variables}
Si une variable apparaît dans un atome de la tête, elle doit également être dans le corps. Chaque instance d'une variable va être groundée par clingo afin de trouver tous les modèles à notre problème.
Prenons un petit exemple, voici un problème encodé avec 2 règles :
\begin{center}
\emph{
	\textcolor{blue}{
  parentOf(jenny, charles).\\
  parentOf(mary, jenny).
	}
}
\end{center}
Ce programme contient 2 faits : Charles est un parent de Jenny, et Jenny est un parent de Mary. C'est ainsi que l'on souhaite comprendre les termes "parentOf(jenny, charles)" et "parentOf(mary, jenny)".
Si on souhaite désormais spécifier que Charles est un grand parent de Mary, on peut le faire en ajoutant le terme : "grandparentOf(mary, charles)". Cependant, ASP est capable de faire de la déduction.
En effet, on peut généraliser la notion de grand parent comme étant le parent d'un parent de la façon suivante :
\begin{center}
\emph{
	\textcolor{blue}{
  grandparentOf(X, Z) :- parentOf(X, Y), parentOf(Y, Z).
	}
}
\end{center}
Pour chaque valeurs possible de X, Y, et Z, le grounding d'ASP va nous créer des règles associées.
Ainsi, les 27 règles dont grandparentOf(mary, mary) :- parentOf(mary, mary), parentOf(mary, mary), ou encore grandparentOf(charles, mary) :- parentOf(charles, jenny), parentOf(jenny, mary) vont être créées 
par cette façon de procéder. Ceci n'est pas du tout un problème : le solveur d'ASP va simplement regarder si les atomes des queues sont vraies ou pas. Si ce n'est pas le cas, comme le faux implique tout, 
les règles seront vraies. Si c'est le cas, la tête sera alors mise à vraie. Sachant qu'une seule de ces 27 possibilités mène à la véracité d'un nouvel atome, on obtiendra en sortie d'ASP les faits suivants :
\begin{center}
\textcolor{blue}{
  SATISFAIBLE\\
	\emph{
  parentOf(jenny, charles)\\
  parentOf(mary, jenny)\\
  grandparentOf(mary, charles)
	}
}
\end{center}
\subsection{Agrégats}
Les agrégats forment le dernier outil d'ASP que j'ai utilisé dans le cadre de mon stage. Il s'agit d'un moyen de sélectionner un certain nombre d'atomes parmi un ensemble, et de les mettre à vrai. Ils se 
structurent suivant l'exemple suivant :
\begin{center}
\textcolor{blue}{
  0 $\left\{ f(A,B):g(A)\right\}$ 2 :- h(B).
}
\end{center}
La \textbf{borne inférieure}, située à gauche des accolades (ici 0), et la \textbf{borne supérieure} (ici 2) définissent l'intervalle discret du nombre d'atomes que le solveur peut mettre à vrai. Si l'une de ces deux bornes est omise, 
ASP comprend qu'il s'agit de la borne maximale (ou minimale) possible (à savoir 0 pour la borne inf et cardinal de l'ensemble défini par les accolades pour la borne sup).\\
Les agrégats permettent de faire des disjonctions de cas.
\subsection{Exemple du sudoku}
Durant les premières semaines de mon stage, j'ai encodé différents jeux de logique en ASP afin de me familiariser avec le langage. Le jeu le plus simple et connu que j'ai encodé a été le sudoku. Dans cette section, je vais 
briévement détailler le fonctionnement de ce code.\\
On codera une grille de sudoku en ASP en utilisant s, une fonction d'arité 3 prenant R, C et V comme argument, où R est la ligne, C la colonne et V la valeur de la case (R,C) dans la grille.
\begin{center}
\textcolor{blue}{
	$val(1..9)$.\\
}
\emph{1..9 est un raccourci pour dire itérer pour les entiers allant de 1 à 9. Ici, on déclare qu'il y a 9 valeurs dans une fonction d'arité 1 : 'val'}\\
\textcolor{blue}{
	$border(1;4;7)$.\\
}
\emph{On déclare également 3 autres faits pour mémoriser où sont situés les bordures de notre grille care le point virgule permet de  déclarer plusieurs atomes en un}\\
\textcolor{blue}{
	$1\left\{ s(R,C,V): val(V) \right\}1$ :- $val(R) ; val(C)$.\\
}
\emph{On ne prend qu'une seule valeur par carré}\\
\textcolor{blue}{
	$1\left\{ s(R,C,V): val(R) \right\}1$ :- $val(C) ; val(V)$.\\
}
\emph{Une valeur ne peut pas apparaître plusieurs fois dans la même colonne}\\
\textcolor{blue}{
	$1\left\{ s(R,C,V): val(C) \right\}1$ :- $val(R) ; val(V)$.\\
}
\emph{Ni dans la même ligne}\\
\textcolor{blue}{
	$1\left\{ s(R,C,V): val(R), val(C), R1<=R, R<=(R1+2), C1<=C, C<=(C1+2)\right\}1$ :- $val(V) ; border(R1) ; border(C1)$.\\
}
\emph{Une valeur ne peut pas apparaître plusieurs fois dans une sous-grille.}\\
\end{center}
Une fois que l'on a fait cela, il ne nous reste plus qu'à instancier notre problème sur une grille. Pour cela, on renseigne s(R,C,V) pour chaque case déjà pré-numérotée de la grille à notre code, et le solveur se chargera 
de nous renvoyer l'intégralité de la grille complétée.

\subsection{Scripting Python}
En ASP, on peut écrire des morceaux de script en Lua ou en Python. Pour cela, il suffit d'ajouter la commande '\#script (mon\_langage)', puis de taper son code dans le langage que l'on a choisi, et de finir le script par la 
commande '\#end.'. Le sccripting admet trois atouts majeurs :
\begin{itemize}
	\item L'appel au fonction en ASP via la commande '@le\_nom\_de\_ma\_fonction(et, ses, arguments)', permettant d'effectuer des calculs sous Python et de mettre des variables à vrai en fonction de ce qui a été trouvé dans ces 
	calculs.
	\item Un contrôle de la résolution permettant le filtrage des ensembles solutions, l'ajout de faits/règles, la demande de grounding et de solving. C'est ce dont je me suis le plus servi durant mon stage.
	\item Une méthode incrémentale efficace lorsque l'on a besoin d'avoir une notion de temporalité ou bien un calcul qui se fait dans un certain ordre : en fonction des résultats trouvés on peut partir sur une autre piste. La 
	notion d'ordre étant totalement inexistante en ASP (les lignes de code peuvent être écrites dans n'importe quel ordre), cela est pratique.
\end{itemize}

\section{Formalismes d'automates et leur dynamique}
Différents modèles existent pour représenter efficacement un système biologique et manipuler sa dynamique : les deux principaux sont les réseaux booléens synchrones de Stuart Kauffman, et les réseaux asynchrones de René Thomas.
Dans le cadre de mon stage je ne parlerai que des réseaux asynchrones, et plus particulièrement des réseaux d'automates asynchrone (AAN) [Folschette et al., 2015, Paulevé, 2016a], qui forment une extension d'une précédente 
structure appelé "Process Hitting"  [Paulevé et al., 2014].\\

\subsection{AAN et leur traduction en ASP}
Un automate $A$, dans le contexte de mon stage, sera défini comme étant un ensemble d'états $q_0, q_1, ..., q_{|A|-1}$ avec des transitions dont les étiquettes seront un ou plusieurs états (qu'on appellera également \emph{niveaux}) 
d'automates externes. Il n'y a donc ni état final, ni état initial. On ne lui donnera pas non plus de mot à lire en entrée puisqu'on va s'intéresser à la dynamique de nos ensembles d'automates. On parlera alors de 
transition locale : $q_i\xrightarrow{l}q_j$symbolisera le fait que l'on peut passer de l'état $q_i$ à l'état $q_j$ si toutes les conditions de $l$ sont vérifiés.\\
Un réseau d'automates asynchrone est un triplet $(\Sigma,S,T)$ avec :
\begin{itemize}
	\item $\Sigma=\left\{a,b,...\right\}$ est un ensemble fini d'automates non vides.
	\item Si $C_a$ est le nombre d'états d'un automate $a$, alors $S_a=\left\{a_0,a_1,...,a_{C_a-1}\right\}$ est l'ensemble des \textbf{états locaux} de l'automate $a$. $S=\displaystyle{\prod_{a\in\Sigma}}S_a$ est l'ensemble 
	fini des \textbf{états globaux}, et $LS=\displaystyle{\bigcup_{a\in\Sigma}}S_a$ représente l'ensemble de tous les états locaux.
	\item Pour chaque $a\in\Sigma$, $T_a\subseteq\left\{a_i\xrightarrow{l}a_j\in S_a\times\rho(LS/S_a)\times S_a|a_i\neq a_j\right\}$ est l'ensemble des \textbf{transitions locales} d'un automate $a$, avec 
	$\rho$ qui désigne la puissance ensembliste. $T=\displaystyle{\bigcup_{a\in\Sigma}}T_a$ est l'ensemble des transitions locales du modèle.
\end{itemize}
\textbf{Exemple :} On représente l'AAN suivant de cette manière :
\begin{itemize}
	\item $\Sigma =\left\{a,b,c\right\}$
	\item $S_a=\left\{a_0,a_1,a_2\right\}$, $S_b=\left\{b_0,b_1\right\}$ et $S_c=\left\{c_0,c_1,c_2\right\}$
	\item $T_a = \left\{a_0\xrightarrow{b_0}a_1,a_0\xrightarrow{b_1,c_1}a_1,a_1\xrightarrow{b_1}a_0,a_1\xrightarrow{b_0}a_2,a_2\xrightarrow{b_1}a_1\right\}$\\
	$T_b=\left\{b_0\xrightarrow{c_0}b_1,b_1\xrightarrow{a_2}b_0\right\}$\\
	$T_c=\left\{c_0\xrightarrow{b_1}c_1,c_0\xrightarrow{a_2}c_2,c_1\xrightarrow{b_0}c_0,c_1\xrightarrow{a_1}c_2,c_2\xrightarrow{b_1}c_0\right\}$
\end{itemize}
\begin{center}
\begin{tikzpicture}[apdotsimple/.style={apdot}]

  \TSort{(0,2)}{a}{3}{r}
  \TSort{(2.5,0)}{b}{2}{r}
  \TSort{(5,2)}{c}{3}{r}

  \path[local transitions]
    (a_0) edge node[auto] {$b_0$} (a_1)
    (a_1) edge node[auto] {$b_0$} (a_2)
    (a_2) edge node[auto] {$b_1$} (a_1)
    (a_1) edge node[auto] {$b_1$} (a_0)

    (b_0) edge node[auto] {$c_0$} (b_1)
    (b_1) edge node[auto] {$a_2$} (b_0)

    (c_0) edge node[auto] {$b_1$} (c_1)
    (c_1) edge node[auto] {$b_0$} (c_0)
    (c_1) edge node[auto] {$a_1$} (c_2)
    ;
  \path[local transitions, bend left = 105]
      (a_0) edge node[auto] {$b_1, c_1$} (a_1)
    ;
  \path[local transitions, bend left = 90]
      (c_0) edge node[auto] {$a_2$} (c_2)
    ;
  \path[local transitions, bend left = 90]
      (c_2) edge node[auto] {$b_1$} (c_0)
    ;

  \TState{a_0, b_1, c_1}
\end{tikzpicture}
\end{center}
En ASP, on définira un AAN en deux temps :
\begin{itemize}
	\item On commencera par déclarer chacun de nos automates avec les niveaux (= états) qu'il contient :
	\begin {center}
	\emph{
		\textcolor{blue}{
		automaton\_level("a", 0..2).\\
		automaton\_level("b", 0..1).\\
		automaton\_level("c", 0..2).\\
		}
	}
	\end{center}
	\item Enfin, les transitions seront encodés via des labels (t1, t2, ...), et on donnera chacune des conditions, ainsi que l'état d'arrivée, via un fait :
	\begin {center}
	\emph{
		\textcolor{blue}{
		condition(t1, "a", 0). target(t1, "a", 1). condition(t1, "b", 0).\\
		condition(t2, "a", 1). target(t2, "a", 2). condition(t2, "b", 0).\\
		condition(t3, "a", 2). target(t3, "a", 1). condition(t3, "b", 1).\\
		condition(t4, "a", 1). target(t4, "a", 0). condition(t4, "b", 1).\\
		condition(t5, "b", 0). target(t5, "b", 1). condition(t5, "c", 0).\\
		condition(t6, "b", 1). target(t6, "b", 0). condition(t6, "a", 2).\\
		condition(t7, "c", 0). target(t7, "c", 1). condition(t7, "b", 1).\\
		condition(t8, "c", 1). target(t8, "c", 0). condition(t8, "b", 0).\\
		condition(t9, "c", 1). target(t9, "c", 2). condition(t9, "a", 1).\\
		condition(t10, "c", 0). target(t10, "c", 2). condition(t10, "a", 2).\\
		condition(t11, "c", 2). target(t11, "c", 0). condition(t11, "b", 1).\\
		condition(t12, "a", 0). target(t12, "a", 1). condition(t12, "b", 1). condition(t12, "c", 1).\\
		}
	}
	\end{center}
\end{itemize}
Nous avons ainsi défini l'automate de notre \textbf{exemple} en ASP.

\subsection{Sémantiques}
Soit $R = (\Sigma,S,T)$ un AAN (défini ci-dessus). On définit la notion de transition jouable:
\begin{itemize}
	\item On dit qu'une transition locale est \textbf{jouable} si toutes les conditions de celle-ci sont vérifiés\\
	On notera $P_{\zeta}$ l'ensemble des transitions locales jouables depuis un état global $\zeta$\\
	\emph{L'état global $(a_0,b_1,c_1)$ de l'\textbf{exemple} admet une seule transition jouable : $a_0\xrightarrow{b_1,c_1} a_1$}
\end{itemize}
La dynamique d'un AAN se définit à l'aide de sa \textbf{sémantique}. La sémantique forme l'ensemble des propriétés définissant les \textbf{transitions globales jouables} : dans un AAN on s'intéresse à l'évolution globale du réseau.
\subsubsection{Asynchrone}
Soit $R = (\Sigma,S,T)$ un AAN, et $\zeta\in S$ un état global. L'ensemble des transitions globales jouables depuis $\zeta$ pour la sémantique \emph{asynchrone} est donnée par :
\begin{center}
	$U^{asyn}(\zeta)=\left\{\left\{a_i\xrightarrow{l} a_j\right\}|a_i\xrightarrow{l} a_j\in P_{\zeta}\right\}$
\end{center}
De manière informelle : chaque transition locale jouable est une transition globale.\\ \\
En ASP, on commence donc par déclarer que deux états globaux sont différents sur un automate à l'aide de la règle suivante :\\
\emph{
1. \textcolor{blue}{different\_on(Gs1, Gs2, Automaton) :-} \% Cet atome est vrai\\
2.\qquad \textcolor{blue}{global\_state(Gs1), global\_state(Gs2), Gs1 != Gs2,} \% si deux états globaux diffèrent\\
3.\qquad \textcolor{blue}{automaton(Automaton),} \% qu'il existe un automate\\
4.\qquad \textcolor{blue}{active\_in\_g(level(Automaton, LevelI), Gs1),} \% sur lequel $Gs1$ vaut $I$\\
5.\qquad \textcolor{blue}{active\_in\_g(level(Automaton, LevelJ), Gs2),} \% et sur lequel $Gs2$ vaut $J$\\
6.\qquad \textcolor{blue}{LevelI != LevelJ.} \% avec $I \neq J$\\ \\
}
Ensuite, on fait comprendre à ASP qu'une transition locale n'est pas jouable si l'une (au moins) de ses conditions n'est pas vérifiée de la façon suivante :\\
\emph{
1. \textcolor{blue}{unplayable(Transition, Gs) :-} \% Cet atome est vrai\\
2.\qquad \textcolor{blue}{local\_transition(Transition),} \% s'il existe une transition locale\\
3.\qquad \textcolor{blue}{global\_state(Gs),} \% un état global\\
4.\qquad \textcolor{blue}{automaton(Automaton),} \% et un automate\\
5.\qquad \textcolor{blue}{active\_in\_g(level(Automaton, LevelI), Gs),} \% sur lequel $Gs$ vaut $I$\\
6.\qquad \textcolor{blue}{condition(Transition, Automaton, LevelJ),} \% alors qu'il devrait valoir $J\neq I$\\
7.\qquad \textcolor{blue}{LevelI != LevelJ.} \% pour que la transition soit jouable\\ \\
}
On déclare ensuite que deux états globaux différent sur au moins un autre automate qu'$Automaton$ avec le terme $not\_equal\_except(Automaton, Gs1, Gs2)$ comme ceci :\\
\emph{
1. \textcolor{blue}{not\_equal\_except(Automaton, Gs1, Gs2) :-} \% Cet atome est vrai\\
2.\qquad \textcolor{blue}{automaton(Automaton), automaton(Automaton2),} \% s'il existe deux automates\\
3.\qquad \textcolor{blue}{Automaton != Automaton2,} \% qui différent\\
4.\qquad \textcolor{blue}{global\_state(Gs1), global\_state(Gs2),} \% deux états globaux\\
5.\qquad \textcolor{blue}{different\_on(Gs1, Gs2, Automaton2).} \% tels que $Gs1 \neq Gs2$ sur Automaton2\\ \\
}
On peut enfin définir ce qu'est une transition globale jouable à l'aide de nos trois précédentes règles :\\
\emph{
1. \textcolor{blue}{playable(Gs1, Gs2) :-} \% Cet atome est vrai\\
2.\qquad \textcolor{blue}{global\_state(Gs1), global\_state(Gs2),} \% s'il existe deux états globaux $Gs1$ et $Gs2$\\
3.\qquad \textcolor{blue}{automaton(Automaton), local\_transition(Transition),} \% un automate et une transition\\
4.\qquad \textcolor{blue}{target(Transition, Automaton, LevelJ),} \% qui fait changer le niveau de l'automate\\
5.\qquad \textcolor{blue}{not unplayable(Transition, Gs1),} \% et avec la transition (locale) qui est jouable\\
6.\qquad \textcolor{blue}{not not\_equal\_except(Automaton, Gs1, Gs2),} \% et $Gs1=Gs2$ excepté sur Automaton\\
7.\qquad \textcolor{blue}{active\_in\_g(level(Automaton, LevelI), Gs1),} \% tel que le niveau d'$Automaton$ dans $Gs1$\\
8.\qquad \textcolor{blue}{active\_in\_g(level(Automaton, LevelJ), Gs2),} \% et celui dans $Gs2$\\
9.\qquad \textcolor{blue}{LevelI != LevelJ.} \% sont différents\\
}
\subsubsection{Synchrone}
Soit $R = (\Sigma,S,T)$ un AAN, et $\zeta\in S$ un état global. L'ensemble des transitions globales jouables depuis $\zeta$ pour la sémantique \emph{synchrone} est donnée par :
\begin{center}
	$U^{syn}(\zeta)=\left\{u\subseteq T|u \neq \emptyset \wedge \forall a\in\Sigma,(P_{\zeta}\bigcap T_a = \emptyset \Rightarrow u\bigcap 
	T_a=\emptyset)\wedge(P_{\zeta}\bigcap T_a\neq\emptyset\Rightarrow |u\bigcap T_a |=1\right\}$
\end{center}
De manière informelle : tous les automates possédant au moins une transition locale jouable doivent changer de niveau.\\ \\
Pour coder cela en ASP, on peut se servir des atomes \emph{\textcolor{blue}{different\_on}} et \emph{\textcolor{blue}{unplayable}}.\\
On ajoute de plus une autre règle : \emph{\textcolor{blue}{has\_playable(Automaton, Gs, LevelI, LevelJ)}} spécifiant que l'on peut faire changer $Automaton$ dans $Gs$ du niveau $I$ vers le niveau $J$.\\
Enfin, il ne nous reste plus qu'a déclarer (par la négation) quelles sont les transitions globales qui ne sont pas jouables. Pour cela, on essaie pour toutes les paires d'états globaux si on peut faire la transition de l'un à 
l'autre en créant un terme \emph{\textcolor{blue}{not\_playable(Gs1,Gs2)}} si la transition ne peut pas avoir lieu. On énumère alors les différents cas de figure :
\begin{enumerate}
	\item si $Automaton$ n'admet aucune transition locale jouable dans $Gs1$ et que $Gs1\neq Gs2$ sur $Automaton$
	\item si $Automaton$ admet une (ou plusieurs) transition locale jouable dans $Gs1$ et qu'il n'existe pas de transition pour faire passer le niveau d'$Automaton$ sur $Gs1$ à celui de $Gs2$
	\item si aucune transition locale n'est jouable depuis $Gs1$
\end{enumerate}
alors la transition $Gs1\rightarrow Gs2$ n'est pas jouable.\\
Enfin, il nous suffit de tester pour toutes les paires d'états globaux distincts si on a \emph{\textcolor{blue}{not\_playable(Gs1,Gs2)}} ou pas. Dans le cas contraire, alors on crée l'atome 
\emph{\textcolor{blue}{playable(Gs1,Gs2)}}.

\subsubsection{Généralisée}
Soit $R = (\Sigma,S,T)$ un AAN, et $\zeta\in S$ un état global. L'ensemble des transitions globales jouables depuis $\zeta$ pour la sémantique \emph{généralisée} est donnée par :
\begin{center}
	$U^{gen}(\zeta)=\left\{u\subseteq T|u \neq \emptyset \wedge \forall a\in\Sigma,(P_{\zeta}\bigcap T_a = \emptyset \Rightarrow u\bigcap 
	T_a=\emptyset)\wedge(P_{\zeta}\bigcap T_a\neq\emptyset\Rightarrow |u\bigcap T_a |\leq 1\right\}$
\end{center}
De manière informelle : tous les automates possédant au moins une transition locale jouable \underline{peuvent} changer de niveau, et au moins l'un d'entre eux doit changer de niveau.\\ \\
Pour coder cela en ASP, on prend exactement le code de la sémantique synchrone, en remplaçant le second cas de figure par ceci :\\ \\
2. si $Gs1$ différe de $Gs2$ sur $Automaton$ et qu'il n'existe pas de transition pour faire passer le niveau d'$Automaton$ sur $Gs1$ à celui de $Gs2$\\ \\
Cela permet bien de ne pas prendre une transition par automate qui est jouable puisqu'on va tester des paires d'états globaux qui ont peut-être 3 automates jouables, et si l'on ne fait qu'une seule transition locale cela ne sera pas 
filtré par notre atome \emph{\textcolor{blue}{not\_playable(Gs1,Gs2)}}.
\subsection{Dynamique}
Soit $U$ une sémantique. On introduit les dernières notions nécessaires pour la définition d'un attracteur :
\begin{itemize}
	\item On appelle \textbf{état stable} un état global ne possédant aucune transition globale jouable pour $U$.\\
	\emph{Dans l'\textbf{exemple}, aucun état global n'est stable pour la sémantique asynchrone (et donc cela est également vrai pour la sémantique synchrone et généralisée)}
	\item Pour $Gs_1,Gs_2\in S$, on notera $Gs_1\rightarrow _UGs_2$ pour symboliser le fait qu'il existe $u\in U(Gs_1)$ tel que si l'on joue toutes les transitions locales de $u$, alors $Gs_1$ devient $Gs_2$.
	\item Un \textbf{chemin} est une famille finie $(A_k)_{1\leq k\leq n}$ et ordonnée d'états globaux tels que pour tout $1\leq k\leq n-1$, $A_k\rightarrow _U A_{k+1}$.
	\item Un \textbf{cycle} est un chemin $(A_k)_{1\leq k\leq n}$ tel que $A_1=A_n$.
	\item Un \textbf{domaine de piège} est un ensemble non vide d'états globaux $S_E\in S$ tel que toute transition globale jouable depuis $S_E$ arrive dans un état global de $S_E$. Plus formellement : 
	$\forall\zeta _1\in S_E \wedge\forall\zeta _2\in S, \zeta _1 \rightarrow _U\zeta _2 \Rightarrow\zeta _2\in S_E$
\end{itemize}
Enfin, on peut définir ce qu'est un attracteur.\\
Un ensemble $A\subseteq S$ d'états globaux, avec $|A|\geq2$ est appellé \textbf{attracteur} si et seulement si c'est un domaine de piège minimal en terme d'inclusion ensembliste.\\
Un travail précédent à démontrer le résultat mathématique suivant dont on va se servir pour trouver les attracteurs au sein d'un AAN :\\ \\
\textbf{Lemme :} Les attracteurs d'un AAN sont exactement les domaines de piège cycliques.

\section{Contribution personnelle}

\section{Conclusion}

\section{Bibliographie}

\end{document}





























